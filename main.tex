\documentclass{article}
\usepackage[a4paper, portrait, margin=2.5cm]{geometry}
\setlength{\parindent}{0pt}
\setlength{\parskip}{20pt}

\begin{document}

\thispagestyle{empty}

\begin{center}
\section*{Find the smallest number $N$ with exactly $K$ factors}
\end{center}

Any number $N$, can be expressed as the product of prime numbers.  
Including some primes that maybe repeated:
$$
N = q_1 \times q_2 \times \cdots \times q_i
$$
For example, if $N = 12$, we get:
$$
12 = 2 \times 2 \times 3
$$
By grouping the repeated primes together we can express this as:
$$
N = p_1 ^ {e_1} \times p_2 ^ {e_2} \times 
\cdots \times p_i ^ {e_i}
$$
Again, if $N = 12$, we get:
$$
N = 2 ^ 2 \times 3 ^ 1
$$
giving:
$$e_1 = 2$$
$$e_2 = 1$$
We can now split each term into two components as follows:
$$
N = 
p_1 ^ {e_1 - n_1} p_1 ^ {n_1} \times 
p_2 ^ {e_2 - n_2} p_2 ^ {n_2} \times 
\cdots \times 
p_i ^ {e_i - n_i} p_i ^ {n_i} 
$$
where each $n_i$ can be any value between $0$ and $e_i$
By rearranging the above into two groups we get:
$$
N = 
( 
p_1 ^ {n_1} \cdot
p_2 ^ {n_2} 
\cdots 
p_i ^ {n_i}
)
\times
(
p_1 ^ {e_1 - n_1} \cdot
p_2 ^ {e_2 - n_2}
\cdots
p_i ^ {e_i - n_i}
)
$$
By selecting different values for the $n_i$'s we can generate many variations of this expression. Each variation produces a pair of terms that multiplies to give $N$ and are therefore factors of $N$.
Moreover, the expression represents ALL possible ways of producing pairs of terms and hence represents ALL factors of $N$.

In our example where $N =12$:
\begin{eqnarray*}
N & = & 12\\
& = & 2 ^ 2 \times 3 ^ 1 \\
& => & e_1 = 2, e_2 = 1
\end{eqnarray*}
we can choose the $n_i$'s in 6 different ways as follows:
\begin{eqnarray*}
12 &=& (2^0 \cdot 3^0) \times (2^2 \cdot 3^1) : n_1 = 0, n_2 = 0 \\
   &=& (1 \cdot 1) \times (4 \cdot 3) \\
   &=& 1 \times 12 \\
12 &=& (2^1 \cdot 3^0) \times (2^1 \cdot 3^1) : n_1 = 1, n_2 = 0\\
   &=& (2 \cdot 1) \times (2 \cdot 3) \\
   &=& 2 \times 6 \\
12 &=& (2^2 \cdot 3^0) \times (2^0 \cdot 3^1) : n_1 = 2, n_2 = 0\\
12 &=& (2^0 \cdot 3^1) \times (2^2 \cdot 3^0) : n_1 = 0, n_2 = 1\\
12 &=& (2^1 \cdot 3^1) \times (2^1 \cdot 3^0) : n_1 = 1, n_2 = 1\\
12 &=& (2^2 \cdot 3^1) \times (2^0 \cdot 3^0) : n_1 = 2, n_2 = 1\\
\end{eqnarray*}
To find how many ways pairs can be constructed for a given $N$ (and therefore given $e_i$'s), we need to determine how many different ways the $n_i$'s can be chosen. 
Since each $n_i$ can be any value from $0$ to $e_i$ or $(e_i + 1)$ distinct values, the total number of combinations to choose all $n_i$'s is:
$$
K = (e_1 +1) \times (e_2 + 1) \times \cdots \times (e_i + 1)
$$
This expression therefore gives the total number of factors of $N$.

Using our example where $N = 12$, $e_1 = 2$ and $e_2 = 1$, we can calculate $K$:
\begin{eqnarray*}
K & = & (e_1 +1) \times (e_2 + 1) \\
  & = & (2 + 1) ( 1 + 1) \\
  & = & 3 \times 2 \\
  & = & 6 \\
\end{eqnarray*}
We can confirm this result by counting the distinct factors of $12$, that we know from brute force to be:
$$
1, 2, 3, 4, 6, 12
$$
and it is indeed 6.

Now that we have an expression that relates exponents of the prime factors of $N$ to the total number of factors, we can use it to find $N$i's for any given $K$ (and in particular the smallest $N$).

First we note that $K$ is itself expressed as a product of terms which means each term is a factor of $K$. 

$K$ may have many factorizations, however as we are looking for the smallest $N$, we want our exponents to be as small has possible. 
Therefore the factorisation we choose will be the prime factorisation of $K$.

For example if $K = 12$ then:
\begin{eqnarray*}
12 
& = & ( e_1 + 1) \times ( e_2 + 1 ) \times (e_3 + 1) \\
& = & 3 \times 2 \times 2 \\
\end{eqnarray*}
therefore:
$$
e_1 = 2, e_2 = 1, e_3 = 1
$$
Plugging these values into our expression for $N$ by pairing the smallest primes with the largest exponents (because we want $N$ to be as small as possible), we get:
\begin{eqnarray*}
N
& = & q_1 ^ {e_1} \times q_2 ^ {e_2} \times q_3 ^ {e_3} \\
& = & 2^2 \times 3^1 \times 5^1 \\
& = & 4 \times 3 \times 5 \\
& = & 60 \\
\end{eqnarray*}

Hence the smallest number with 12 factors is 60. 
This can be verified by brute force (as shown on next page).

We now calculate $N$ for any value of $K$.

\pagebreak
\section*{Numerical Results}
\subsection*{Smallest $N$ with $K$ factors - calculated by brute force}
\begin{verbatim}
   K        N Factors
A( 1)       1 [1]
B( 2)       2 [1, 2]
C( 3)       4 [1, 2, 4]
D( 4)       6 [1, 2, 3, 6]
E( 5)      16 [1, 2, 4, 8, 16]
F( 6)      12 [1, 2, 3, 4, 6, 12]
G( 7)      64 [1, 2, 4, 8, 16, 32, 64]
H( 8)      24 [1, 2, 3, 4, 6, 8, 12, 24]
I( 9)      36 [1, 2, 3, 4, 6, 9, 12, 18, 36]
J(10)      48 [1, 2, 3, 4, 6, 8, 12, 16, 24, 48]
K(11)    1024 [1, 2, 4, 8, 16, 32, 64, 128, 256, 512, 1024]
L(12)      60 [1, 2, 3, 4, 5, 6, 10, 12, 15, 20, 30, 60]
M(13)    4096 [1, 2, 4, 8, 16, 32, 64, 128, 256, 512, 1024, 2048, 4096]
N(14)     192 [1, 2, 3, 4, 6, 8, 12, 16, 24, 32, 48, 64, 96, 192]
O(15)     144 [1, 2, 3, 4, 6, 8, 9, 12, 16, 18, 24, 36, 48, 72, 144]
P(16)     120 [1, 2, 3, 4, 5, 6, 8, 10, 12, 15, 20, 24, 30, 40, 60, 120]
\end{verbatim}

\subsection*{Smallest $N$ with $K$ factors - calculated by factorising $K$}
\begin{verbatim}
A( 1)       1
B( 2)       2
C( 3)       4
D( 4)       6
E( 5)      16
F( 6)      12
G( 7)      64
H( 8)      30
I( 9)      36
J(10)      48
K(11)    1024
L(12)      60
M(13)    4096
N(14)     192
O(15)     144
P(16)     210
Q(17)   65536
R(18)     180
S(19)  262144
T(20)     240
U(21)     576
V(22)    3072
W(23) 4194304
X(24)     420
Y(25)    1296
Z(26)   12288
\end{verbatim}

\pagebreak
\setlength{\parskip}{10pt}
\section*{Python Code}
\subsection*{Smallest $N$ with $K$ factors - calculated by factorising $K$}
\begin{verbatim}
def factorize(n):
    
    factors = [1]
    if n == 1:
        return factors

    for f in range(2,(n)//2 + 1):
        if not n%f:
            factors.append(f)
    factors.append(n)
    return factors

# factorise lots of number and count the factors
numbers = []
for n in range(1,3000):
    factors = factorize(n) 
    numbers.append({
        "N":n,
        "K":len(factors),
        "factors":factors,
    })    

print(f"   K        N Factors")
for k in range(1,27):                  # A - Z
    for n in numbers:
        if n["K"] == k:
            print(f"{chr(64+k)}({k:2}){n['N']:8}", n["factors"])
            break
\end{verbatim}

\section*{smallest $N$ via prime factoring of $K$}
\begin{verbatim}
small_primes = [2, 3, 5, 7, 11, 13, 17, 19, 23]

def prime_factors(n):
    factors = []
    for p in small_primes:
        while not n%p:
            n = n // p
            factors.append(p)
            if n == 1:
                return factors

print(F"{chr(64+1)}({1:2}){1:8}")
for k in range(2,27):
    N = 1
    j = 0
    for factor in reversed(prime_factors(k)):
        N = N * small_primes[j] ** (factor-1)
        j += 1
    print(F"{chr(64+k)}({k:2}){N:8}")
\end{verbatim}
\end{document}
